\documentclass[aps,prb,reprint,showpacs,floatfix,superscriptaddress, onecolumn, nofootinbib, 10pt]{revtex4-2}

\usepackage{amsmath,amsthm,amssymb}
\usepackage{graphicx}% Include figure files
\usepackage{dcolumn}% Align table columns on decimal point
\usepackage{bm}% bold math
\usepackage{color}
\usepackage{epsfig}
\usepackage{multirow}
\usepackage{mathrsfs}
\usepackage{hyperref}
\usepackage{cleveref}
\usepackage{epstopdf}
\usepackage{subfigure}
\usepackage{autobreak}
\usepackage{todonotes}
\usepackage{physics}
\usepackage{bbm}
\usepackage[normalem]{ulem}
\usepackage[margin=0.5cm]{geometry}

\usepackage[absolute,overlay]{textpos}

%Macros for mathematical notations

\newcommand{\V}[1]{\boldsymbol{#1}} %# vector
\newcommand{\M}[1]{\boldsymbol{#1}} %# matrix
\newcommand{\Set}[1]{\mathbb{#1}} %# set
\newcommand{\D}[1]{\Delta#1} %# \D{t} for time step size
\renewcommand{\d}[1]{\delta#1} %# \d{t} for small increment
\newcommand{\av}[1]{\left\langle #1\right\rangle } %take average

\newcommand{\sM}[1]{\M{\mathcal{#1}}} %matrix in mathcal font
\newcommand{\dprime}{\prime\prime} % double prime
%\global\long\def\i{\iota}
%\renewcommand{\i}{\iota} %i for imaginary unit
%\renewcommand{\i}{\mathsf i} %i for imaginary unit
\newcommand{\follows}{\quad\Rightarrow\quad} %=>
\newcommand{\eqd}{\overset{d}{=}} %=^d
\newcommand{\spe}[1]{\mathscr{#1}}  %important quantities in mathscr font
\newcommand{\eps}{\epsilon}

\newcommand{\response}[1]{{\color{black}#1}} % for authors' response
\newcommand{\comment}[1]{{\color{blue}#1}} % for referee's comment


\begin{document}
\preprint{Preprint}

\title{Response to Referee Comments for Manuscript NJP-117075}
\author{Analabha Roy}
\date{\today}

\maketitle

\vspace{1em}

\noindent \textbf{Response to the Referee: 2's comment}
\begin{enumerate}
	\item {\bf Major Comments}
	\begin{enumerate}
		\item The referee comments on, \comment{``System size dependence: The manuscript presents insightful results for a system size
				of N=8; however, a crucial aspect remains unexplored – the dependence of the
				findings on the system size. How do the observed dynamics and stability vary with an
				increase in system size?"}\\
		
		\response{
			We thank the referee for this suggestion. 
			We observed that at thermodynamic limit it is observed that the DMBL is unstable. This result differs from the behavior of short-range models described in previous studies, where a non-analytic transition occurs from a thermal phase to a strictly local phase at finite driving frequencies. We have revised this discussion in the manuscript and updated the Introduction and Conclusion part.
		}
	
		\item The referee comments on, \comment{``Influence of rotational error $\epsilon_B$: The manuscript effectively investigates the
			impact of $\epsilon_A$={0.03,0.05,0.1} on the chimera states, leaving the role of $\epsilon_B$, set to 0.9, less discussed. How does varying $\epsilon_B$ impact the chimera
			states?"}\\
		\item The referee comments on, \comment{``\underline{Higher root of Bessel function and stability of chimera states:}"}\\
		\begin{enumerate}
			\item \comment{``The paper introduces a higher root of the Bessel function without explicitly justifying its significance. The importance of this higher root and its relevance to the stability
				of chimera states need clarification. How does the selection of a higher root impact the system's behavior, and why is it crucial for the observed dynamics?"}
			\item \comment{``Moreover, the statement “The stability of the chimera order diminishes even if there is a minor deviation from the CDT/DL point” (page 15, lines 43-44) implies that chimera states might not be stable under slight deviations from the CDT/DL point. This raises a critical question regarding the practical implementation of creating chimera states in experiments. To address this, it would be valuable for the paper to explore potential techniques or strategies aimed at stabilizing the DMBL part of the chain. Elaborating on practical considerations and potential solutions would enhance the paper's applicability and contribute to a more comprehensive understanding of the proposed model."}
		\end{enumerate}
		\item The referee comments on, \comment{``\underline{DTC phase stability and entanglement entropy:}"}\\
		\begin{enumerate}
			\item \comment{``The paper seems to be intended for an audience from DTC and MBL/DMBL fields. However, for broader accessibility and comprehension, a brief introductory paragraph about realization and fundamental aspects of DTC within DMBL
				systems would be beneficial. How is DTC defined in these systems, and what are its fundamental properties? Additionally, how is $\omega$ chosen in relation to the
				periodicity of the Hamiltonian? Moreover, a preliminary explanation of how analyzing the time evolution of local magnetization and its FFT contributes to defining DTC would greatly enhance reader comprehension."}
			\item \comment{``It is important to clarify the results shown in Fig. 7, where the behavior of regional magnetization might confuse new readers in the DTC field. This figure suggests that regional magnetization decreases over time under strong coupling and all-to-all interactions, which at the first glance seems to contradict the main text’s assertion that these conditions correspond to stable chimera states. Therefore, a comment  explaining how the magnetization relates to FFT based on the results would be helpful for correctly understanding the results."}
			\item \comment{``There appears to be inconsistency in the analysis of EE and regional magnetisation along with FFT. EE is shown for almost two times longer dynamics that regional magnetization and FFT. Could the authors provide a comparison of regional  magnetization and its FFT for the extended time frames considered for entanglement entropy?"}
		\end{enumerate}
		\item The referee comments on, \comment{``\underline{Applications of chimera state:} The introduction of Section 4 briefly touches upon applications of chimera states, but it lacks depth. How can the chimera state findings be applied in practice? Expanding on potential applications and providing references to existing literature will be beneficial. This would enrich the discussion and highlight the practical relevance of their findings."}
	\end{enumerate}

	\item[] {\bf Minor Comments}
	\begin{enumerate}
		\item The referee says, \comment{``\underline{Page 6, lines 30-31 and page 18, lines 47-48:} please add a reference to the Baker-Campbell-Hausdorff formula, e.g. [67] as in the Appendix B, page 22, lines 44-45"}\\
		\item The referee says, \comment{``\underline{Page 6, lines 46-48:} clarify what is $\mathcal{J}_0$, e.g. “of the higher roots of zeroth-order Bessel function $\mathcal{J}_0$”"}\\
		\item The referee comments on, \comment{``\underline{Page 8, caption of Fig. 3:}"}
		\begin{enumerate}
			\item The referee says, \comment{``refine the caption for clarity: “plotted for different values of amplitude h of the periodic drive. The x-coordinate plots 4h/$\omega$, where drive frequency is kept constant $\omega$=20..."}\\
			\item The referee says, \comment{``The first such point is shown...” $\rightarrow$ “The first two points are shown..."}
		\end{enumerate}
		\item The referee comments on, \comment{``\underline{Page 9, caption of Fig. 4:}"}
		\begin{enumerate}
			\item The referee says, \comment{``This is the suggestion of a notation change for smoother reading: “Site(i)” → “i”, 
			e.g. “for each i-th spin at region A (i=0,1,2,3) and region B (i=4,5,6,7)...”. To implement this modification consistently: change the x-coordinate labels in the Fig. 4 from “Site(i)” → “i”, and update accordingly the main text/ figures/ captions to
			maintain consistency"}\\
			\item The referee says, \comment{``Revise “spin coupling ($J_0$=0.027/T)” $\rightarrow$ “spin coupling ($J_0$=0.072/T)”"}
			\item The referee says, \comment{``Please add also the information for which root of the Bessel function the plot is obtained."}
		\end{enumerate}
		\item The referee comments on, \comment{``\underline{Page 10, Fig. 5:}"}
		\begin{enumerate}
			\item The referee says, \comment{``Define $M_A^z$, which appears in the y-label, in the main text or the caption for
				clarity. The formal introduction of regional magnetization occurs in Section 4, while
				Fig. 5 is situated within Section 3."}\\
			\item The referee says, \comment{``Specify a CDT/DL point, i.e. to which root of the Bessel function it corresponds"}
		\end{enumerate}
		\item The referee comments on, \comment{``\underline{Page 11, lines 50-51:} Specify “a CDT/DL point...” while throughout the paper the root of the Bessel function is changing, e.g. Fig. 4 and Fig. 5"}
		\item The referee comments on, \comment{``\underline{Page 13, Fig 8:} Correct the notation “$\omega/\omega_D$” → “$\Omega/\omega$”"}
		\item The referee says, \comment{``Ensure all paper captions are reviewed for consistency. Additionally, include all relevant
			parameters in the captions that would enable interested readers to reproduce your
			results effectively."}
	\end{enumerate}

	\newpage
	\noindent \textbf{Response to the Referee: 3's comment}
	\begin{enumerate}
		\item The referee says, \comment{``My main concern is regarding the context in which the word “chimera” has been used
			in this work. The word chimera was originally used for the co-existence of synchronized and unsynchronized dynamics of coupled “identical” oscillators in the “Phys. Rev. Lett. 93, 174102 (2004)” which was initially discovered in the following
			works “Y. Kuramoto and D. Battogtokh, Nonlinear Phenom. Complex Syst. 5, 380 (2002), S. I. Shima and Y. Kuramoto, Phys. Rev. E 69, 036213 (2004)”. The surprise that led to the discovery of the chimera state resided in the fact that all the
			subsystems were identical and were subjected to the same environment but behaved differently only due to different initial configurations. But in this and the earlier work mentioned by authors “Phys. Rev. Lett. 126, 120606 (2021)”, regions A and B are under different drive conditions. Hence, the spins in these regions are not under identical environments. In such a configuration, it is trivial that both regions can behave differently since they are subjected to different drive conditions. Hence, this co-existence of different phases under inhomogeneous drive conditions for different
			regions does not fall under the novel phenomenon of “chimera”.\\
			
			I understand that this issue arises due to the fact that the related work “Phys. Rev. Lett. 126, 120606 (2021)” has referred to this phenomenon as a “chimera state”, even though one of the authors from the same paper has used the definition of the identical oscillator in their earlier work “Phys. Rev. E 92, 062924 (2015)”. Therefore, I request the authors either remove the word chimera completely or use the word “chimeralike state” which has been used in the work “Phys. Rev. E 103, 012214 (2021)” where authors have discovered a chimeralike state in almost-identical oscillators. Along with this, authors should provide a clear distinction that the definition of “chimeralike state” used in this work differs from the definition involving identical oscillators and follows from the earlier work “Phys. Rev. Lett. 126, 120606 (2021)” and “Phys. Rev. E 103, 012214” which involve non-identical systems. I
			suppose this is crucial to avoid misunderstandings and further confusion about the novel “Chimera State” for the readers of this reputed journal."}\\
		
		\item The referee says, \comment{``Adding to the previous point on chimera-like states, it would be interesting to see if
			the chimera state emerges even for almost the same drive conditions for regions A
			and B, i.e. for $\epsilon A \approx \epsilon B$ where they only differ by a small value."}\\
		\item The referee says, \comment{``Authors can also investigate the onset of the chimera state as a function of $\epsilon A$ - $\epsilon B$."}\\
		\item The referee says, \comment{``To avoid confusion, the 45th line of Page 2 should be modified to indicate that the concept of “time crystal” was introduced by Frank Wilczek, not just the Discrete time crystals."}\\
		\item The referee says, \comment{``$J_0$ is not defined after equation 7 in the main text."}\\
		\item The referee says, \comment{``Recent works have not been included in the manuscript, I list some of the recent work on chimera states in time crystals and observation of discrete-time crystals for the consideration of authors:."}
		\begin{enumerate}
			\item \comment{``Observation of a Dissipative Time Crystal”, Phys. Rev. Lett. 127, 043602 (2021)"}
			\item \comment{``Observation of a Prethermal U(1) Discrete Time Crystal”, Phys. Rev. X 13, 041016 (2023)"}
			\item \comment{``Observation of time crystal comb in a driven-dissipative system”, arXiv:2402.13112 (2024)}
			\item \comment{``Exotic synchronization in continuous time crystals outside the symmetric subspace”, arXiv:2401.00675 (2024)"}
		\end{enumerate}
	\end{enumerate}
\end{enumerate}
		
		

	
	
\noindent \textbf{Summary of important changes to the  manuscript}
\begin{enumerate}
	\item write the changes you have made$\dots$.
\end{enumerate}
\end{document}